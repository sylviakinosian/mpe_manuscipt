\documentclass[12pt]{article}
\renewcommand{\baselinestretch}{1.5} 

\setlength{\parskip}{1ex plus0.5ex minus0.5ex} 
\setlength{\textheight}{9.0in}
\setlength{\textwidth}{6.5in}
\setlength{\oddsidemargin}{0in}
\addtolength{\topmargin}{-.875in}
\usepackage{caption}
\usepackage{subcaption}
\usepackage{float}
\usepackage{rotating}
\usepackage{mathtools}
%\usepackage{natbib}
\usepackage{epigraph}
\usepackage{graphicx}
\usepackage{longtable}
\usepackage[colorinlistoftodos]{todonotes}

% An example of how to make your life easier for odd formatting
\usepackage{xspace}
\newcommand{\structure}{\small{STRUCTURE}\xspace}

% bib stuff from Will
%%%%%%%%%%%%%%%%%%%%%%%%%%%%%%%%%%%%%%%%%%%%%%
\usepackage[citestyle=authoryear,bibstyle=authoryear,sorting=nyt,maxcitenames=2,maxbibnames=10,minbibnames=6,doi=true,url=false,isbn=false,firstinits=true,uniquename=false,uniquelist=false]{biblatex}
\addbibresource{bib.bib}
\renewbibmacro*{name:andothers}{% Based on name:andothers from biblatex.def
  \ifboolexpr{
    test {\ifnumequal{\value{listcount}}{\value{liststop}}}
    and
    test \ifmorenames
  }
    {\ifnumgreater{\value{liststop}}{1}
       {\finalandcomma}
       {}%
     \andothersdelim\bibstring[]{andothers}}
    {}}
\renewcommand*{\finalnamedelim}{%
  \ifnumgreater{\value{liststop}}{2}{\finalandcomma}{}%
  \addspace and\space}
\renewbibmacro{in:}{}
\AtEveryBibitem{%
  \clearfield{day}%
  \clearfield{month}%
  \clearfield{endday}%
  \clearfield{endmonth}%
  \clearfield{language}%
  \clearfield{number}%
}
\DeclareFieldFormat[article]{citetitle}{#1}
\DeclareFieldFormat[article]{title}{#1}
\DeclareFieldFormat[article]{pages}{#1}
\DeclareNameAlias{sortname}{last-first}

% no parentheses around year in bibliography
\makeatletter
\ifcsundef{abx@macro@\detokenize{date+extrayear}}{%
  }{%
  \renewbibmacro*{date+extrayear}{%
    \iffieldundef{year}{%
      }{%
      \addperiod\space
      \printtext{\printdateextra}%
    }%
  }%
}
\makeatother
%%%%%%%%%%%%%%%%%%%%%%%%%%%%%%%%%%%%%%%%%%%%%%%%%%

% WDP: Something to make track changes easier to track
\usepackage{changes}
\setdeletedmarkup{\textcolor{red}{\sout{#1}}}
%\usepackage[final]{changes} % - to see the "final" version
%%%%%%%%%%%%%%%%%%%%%%%%%%%%%%%%%%%%%%%%%%%%%%%%%%%%%%%%%%%%%%

\begin{document}
\begin{flushleft}
{\large{Cryptic diversity in the model fern genus \textit{Ceratopteris} (Pteridaceae)}}

Sylvia P. Kinosian\textsuperscript{a,}*, William D. Pearse\textsuperscript{a}, and Paul G. Wolf\textsuperscript{b}

{\small{\textsuperscript{a} Ecology Center and Department of Biology, Utah State University, Logan, Utah, USA, 84322\\
\textsuperscript{b} Department of Biological Sciences, University of Alabama in Huntsville, Hunstville, Alabama, USA, 35899}}

\textbf{Abstract}

Cryptic species are present throughout the tree of life. They are especially prevalent in ferns, because of processes such hybridization, polyploidy, and reticulate evolution. In addition, the simple morphology of ferns limits phenotypic variation and makes it difficult to detect cryptic species. The model fern genus \textit{Ceratopteris} has long been suspected to harbor cryptic diversity, specifically in the highly polymorphic \textit{C. thalictroides}. Yet no studies have included samples from throughout its pan-tropical range or utilized genomic sequencing, making it difficult to assess the full extent of cryptic variation within this genus. Here, we present the first multilocus phylogeny of the genus using reduced representation genomic sequencing (RADseq), and examine population structure, phylogenetic relationships, and ploidy level variation. We recover similar species relationships found in previous studies, find support for a named cryptic species as genetically distinct
% WDP: "named cryptic species" is a bit ambiguous - name it?...
, and identify a novel genomic variation within two of the mostly broadly distributed species in the genus.

Keywords: C-fern, polyploid, systematics, RADseq

* Corresponding author at: 5305 Old Main Hill, Logan, Utah, USA, 84322. \\ \textit{E-mail address:} sylvia.kinosian@gmail.com (S. P. Kinosian).

\setlength{\parindent}{5ex}

\vspace{30cm}
\end{flushleft}
\begin{flushleft}
{\large\textbf{1. Introduction}}\\

Describing the formation of species is essential to understanding the patterns and processes that create biodiversity. Detecting differences among species becomes increasingly challenging when taxa are morphologically similar, but genetically distinct and reproductively isolated \autocite{Bickford2007, Masuyama1992, Paris1989}. Such cryptic species have often been historically described as one larger species, or species complex, due to morphological similarities \autocite{Paris1989}. However, advances in molecular methods have revealed that cryptic species can be monophyletic with disparate morphologies, ecological niches and functions \autocite{Amato2007, Hebert2004, Sattler2007, Southgate2019}. In addition, cryptic species complexes can also be paraphyletic and separated by considerable evolutionary time, yet look similar and occupy comparable niches \autocite{Amor2014, Cunnington2005}. 

Despite the recent shift in molecular approaches to investigate cryptic species, much is still unknown about their evolutionary history and ecosystem functions. Cryptic species complexes often have very broad distributions \autocite{Der2009, Knowlton1993, Nygren2014} and can occur in sympatry, allopatry, or parapatry \autocite{Bickford2007}, making adequate sampling challenging. Studies have shown that cryptic species are prevalent across the Tree of Life (e.g. \cite{Bickford2007, Del_Carmen_Molina2011, Hebert2004, Nygren2014}), and ferns in particular are known to harbor many lineages with cryptic variation \autocite{Adjie2007, Paris1989, Yatabe2009}. However, deciphering species boundaries in ferns can be quite difficult due to their ease of dispersal via spores, making gene flow possible across vast distances \autocite{Barrington1993, Tryon1970}. Compared to seed plants, ferns have a high prevalence of polyploidy and reticulate evolution \autocite{Barrington1989, Paris1989, Sigel2016, Otto2000, Otto2000}. In addition, changes in ploidy can contribute to cryptic variation in ferns by altering niche space or offspring viability \autocite{Otto2007, Southgate2019, Masuyama2002}, yet have small effects on phenotype \autocite{Patel2019}. The widespread fern genus \textit{Ceratopteris} Brong. (Pteridaceae) is understood to have considerable cryptic variation, hybridization and polyploidy \autocite{Adjie2007, LloydTax1974, Masuyama2010}, making it an ideal system to investigate the origins and ecology of cryptic species.

The fern genus \textit{Ceratopteris} is a pan-tropical aquatic clade consisting of seven named species, three of which are cryptic and tetraploid \autocite{PPGI, LloydTax1974, Masuyama2010}. \textit{Ceratopteris} is perhaps best known for the model organism \textit{C. richardii}, which has been used as such since the late twentieth century \autocite{Banks1994, hickock1987, Hickok1995}. Sometimes called the ``\textit{Arabidopsis} of the fern world'' \autocite{Sessa2014}, \textit{C. richardii} is an ideal model system because of its fast life cycle and ease of cultivation \autocite{hickock1987}; in addition, it can be transformed with recombinant DNA \autocite{Muthukumar2013, Plackett2014}, has a reference ontogeny framework \autocite{Conway2019-cd}, and is currently the only homosporous fern to have a published genome sequence \autocite{Marchant2019}. In addition to \textit{C. richardii}, other \textit{Ceratopteris} species have been studied in a lab environment \autocite{hickok1974, Hickok1977}, and have potential for further research. The species boundaries within the genus are blurry, however, and there is evidence that all species in the group (both diploid and tetraploid) hybridize to some extent \autocite{Adjie2007, hickok1974, Hickok1977, Hickok1979, LloydTax1974}. In such a well-utilized model genus, there is still a need to better understand species boundaries, evolutionary history, and occurrences of cryptic species. In particular, we need to understand the evolutionary dynamics of the \textit{Ceratopteris} genus as a whole in order to best utilize this model system for future work. 

The first and only comprehensive monograph for \textit{Ceratopteris} was written by Lloyd in 1974, and employed a matrix of morphological traits to identify four species in the genus\nocite{LloydTax1974}. However, \textit{Ceratopteris} is a notorious group in terms of morphology: there are relatively few informative physical characters \autocite{LloydTax1974}; in addition, habitat and developmental stage can have a large effect on plant phenotype \autocite{Masuyama1992}. Most challenging for species delimitation is that rampant hybridization in the genus is known to further alter morphology and so make field identification difficult \autocite{hickok1974, LloydTax1974, Masuyama2010}. While Lloyd named four species, he noted that \textit{C. thalictroides}, the most widespread species in the genus, is ``highly polymorphic'' \autocite{LloydTax1974} and is likely a cryptic species complex. In response, Masuyama and colleagues conducted a series of studies on Asian plants under the name of \textit{C. thalictroides}, examining chloroplast DNA and cross-breeding \autocite{Masuyama2002}, cytological characteristics \autocite{Masuyama2005}, morphological traits \autocite{Masuyama1992, Masuyama2008}, and nuclear DNA \autocite{Adjie2007}. As a result, three cryptic species and two varieties were named from entities originally described as \textit{C. thalictroides} \autocite{Masuyama2010}. Molecular evidence suggests that all three cryptic species of \textit{C. thalictroides} have independent hybrid allopolyploid origins and are paraphyletic \autocite{Adjie2007}, the latter providing the most compelling evidence that they cannot be described under the same name. However, that study included only one nuclear marker; multilocus analyses are generally needed to substantiate paraphyly and polyploid origins \autocite{Eaton2013, Jorgensen2017}, and particularly in a case as challenging as \textit{C. thalictroides}. In addition, while cryptic species in \textit{Ceratopteris} have been investigated in the Old World, there is a similar problem of morphological variability in New World \textit{C. thalictroides} \autocite{Masuyama2010, LloydTax1974}, indicating a potential for more cryptic species to be discovered.

As a well-studied, pan-tropical group known to include cryptic species, \textit{Ceratopteris}, and in particular \textit{C. thalictroides}, is an ideal system in which to study the process of cryptic speciation. In this study, we produce the first multilocus genomic analysis of \textit{Ceratopteris} using restriction-site associated DNA sequencing (RADseq). RADseq is a cost-effective way to generate a large amount of genomic data, and many downstream analysis tools are currently available. We apply three approaches to identify cryptic species. First, we estimate population structure and hybridization; second, we reconstruct phylogenetic relationships among samples; finally, we investigate ploidy levels across individuals. This study is a step towards resolving species boundaries within this convoluted genus. Our work will also provide a phylogenetic reference for future studies on \textit{Ceratopteris}, as well as find areas within the genus in need of additional research.

\vspace{1cm}

{\large\textbf{2. Materials and methods}}

To best analyze the cryptic diversity within \textit{Ceratopteris} we gathered samples from across its pan-tropical distribution. We utilized RADseq to generate a large genomic dataset, which was processed using the \textit{ipyrad} pipeline \autocite{Eaton2020} for downstream analysis. Our three-part approach for assessing cryptic diversity takes advantage of the flexibility of RADseq data, and provides a unique window into the processes of speciation and diversification in a complex genus. 

All parameter values and code for data processing and downstream analyses can be found on GitHub (github.com/sylviakinosian/ceratopteris\_RADseq).

\textbf{2.1. Taxon sampling}

Because of its broad distribution, we chose to use a combination of herbarium and silica-dried material from field collections to sample \textit{Ceratopteris} (Fig. \ref{map}). For all herbarium specimens included, we assigned them to species according to the labels provided by the collector. For the samples collected in the field from Costa Rica, Taiwan, and Australia we utilized the keys provided by \textcite{LloydTax1974} and \textcite{Masuyama2010} to identify each to species or subspecies, where applicable. 

We collected 90 samples, 56 from herbarium specimens and 34 from silica-dried tissue. These samples represented five of seven named species of \textit{Ceratopteris}, and one sample of \textit{Acrostichum aureum} from Australia for use as an outgroup. Included in our silica-dried specimens was a sample of the common lab strain of \textit{C. richardii} (\textit{Hnn}). The two missing species of \textit{Ceratopteris} were cryptic species of \textit{C. thalictroides} (L.) Brongn. \autocite{Masuyama2010}: \textit{C. oblongiloba} Masuyama \& Watano, known from Southeast Asia; and \textit{C. froesii} Brade, endemic to Brazil. Both have relatively narrow ranges and few herbarium species, especially \textit{C. froesii}. We were unable to obtain any material from herbarium specimens labeled as either of these species, and none of the plants collected in the field keyed out to \textit{C. oblongiloba} or \textit{C. froesii}. 

Herbarium specimens were chosen based on age (less than about 30 years old) and color (leaves still green). These specimens were collected from the Harvard University Herbaria (HUH), the Steere Herbarium at the New York Botanic Garden (NY), the University of California, Berkeley (UC), the United States National Herbarium at the Smithsonian (US), and the Pringle Herbarum at the University of Vermont (VT). Fresh tissue collections were obtained from Taiwan, China, Costa Rica, and Australia. All field collections were stored on silica gel and vouchers were deposited at the Intermountain Herbarium (UTC) and James Cook University (JCT). See Supplementary data 1 for full details of specimen sources.

\textbf{2.2. DNA extraction}

Silica-dried plant tissue samples were supplied to the University of Wisconsin-Madison Biotechnology Center. DNA was extracted using the QIAGEN DNeasy mericon 96 QIAcube HT Kit, and quantified using the Quant-iT\textsuperscript{TM} PicoGreen\textsuperscript{\textregistered} dsDNA kit (Life Technologies, Grand Island, NY). 17 specimens were extracted using a modified CTAB method \autocite{Doyle1987} by SPK at Utah State University. These specimens were send to the University of Wisconsin-Madison Biotechnology Center, analyzed for quality and then pooled with the rest of the samples.

\textbf{2.3. Library construction and sequencing}

Libraries were prepared following Elshire \textit{et al.} (2011)\nocite{Elshire2011} with minimal modification; in short, 100 ng of DNA was digested using PstI and BfaI (New England Biolabs, Ipswich, MA) after which barcoded adapters amenable to Illumina sequencing were added by ligation with T4 ligase (New England Biolabs, Ipswich, MA). 96 adapter-ligated samples were pooled and amplified to provide library quantities amenable for sequencing, and adapter dimers were removed by SPRI bead purification. Quality and quantity of the finished libraries were assessed using the Agilent Bioanalyzer High Sensitivity Chip (Agilent Technologies, Inc., Santa Clara, CA) and Qubit\textsuperscript{\textregistered} dsDNA HS Assay Kit (Life Technologies, Grand Island, NY), respectively. A size selection was performed to obtain 300 - 450 BP fragments. Sequencing was done on Illumina NovaSeq 6000 2x150 S2. Images were analyzed using the standard Illumina Pipeline, version 1.8.2. 

\textbf{2.4. Data processing}

Raw data were demultiplexed using stacks v. 2.4 \autocite{Catchen2011, Catchen2013} process\_radtags, allowing for a maximum of one mismatch per barcode. Demultiplexed FASTQ files were paired merged using \textit{ipyrad} version 0.9.52 \autocite{Eaton2020}. Low quality bases, plus adapters and primers were removed from each read. Filtered reads were clustered at the default setting of 85\%, and we required a sequencing depth of 6 or greater per base and a minimum of 30 samples per locus to be included in the final assembly. The \textit{ipyrad} pipeline defines a locus as a short sequence present across samples. From each locus, \textit{ipyrad} identifies single nucleotide polymorphisms (SNPs); these SNPs are the variation used in downstream analyses. Since stacks and \textit{ipyrad} assume diploidy, we processed all samples as such. It is well-known that some species of \textit{Ceratopteris} are non-diploid \autocite{Adjie2007, Masuyama2010}, and so we explore the ploidy of each sample more thoroughly later in our analyses.

Our initial sampling contained 89 specimens of \textit{Ceratopteris} and one sample of \textit{Acrostichum aureum}, plus six replicates. After checking for quality and similarity of replicates, they were combined with the corresponding samples. Of the 90 unique starting samples of \textit{Ceratopteris}, 26 either extracted very poorly or yielded few loci (less than 1000 loci per individual). These were removed from downstream analysis. We also removed 13 samples from Australia, an area we were able to sample on a very fine spatial scale: one individual per Australian population was included in the final dataset so we did not over represent Australian specimens in our analyses. Our final data set included 50 samples of \textit{Ceratopteris} and one sample of \textit{A. aureum}.  We did two separate runs of the \textit{ipyrad} pipeline: one with \textit{A. aureum} for phylogenetic analysis, and one without for ploidy estimation and population structure analysis. Because \textit{A. aureum} is sister to \textit{Ceratopteris} and separated by over 30 million years \autocite{PPGI}, the total number of SNPs retrieved for this individual was quite low.

\textbf{2.5. Population and genomic structure analysis}
% WDP: See top of MS for how \structure works; I sense you might like this trick
Because species of \textit{Ceratopteris} are known to hybridize \autocite{hickok1974}, but also form reproductively isolated cryptic species \autocite{Masuyama2002}, we wanted to examine population structure of the genus across its distribution. The program \structure v. 2.3.4 \autocite{Pritchard2000} is designed to determine admixture (hybridization) between populations of one or several closely related taxa. The program assumes that each individual's genome is a mosaic from \textit{K} source populations and uses genotype assignments from SNPs to infer population structure and admixture. We utilized a set of 8478 SNPs, created in \textit{pyrad} by selecting one SNP per loci. We ran \structure for \textit{K} = 2 - 6 with 50 chains for each \textit{K}. We then used {\small{CLUMPAK}} \autocite{Kopelman2015} to process the \structure output and estimate the best \textit{K} values \autocite{Evanno2005, Pritchard2000}.

\textbf{2.6. Species tree inference}

To construct a phylogeny for the samples of \textit{Ceratopteris}, we used the program {\small{TETRAD}}, which is included in the \textit{ipyrad} analysis toolkit and is based on the software \small{SVDQuartets} \autocite{Chifman2015}. {\small{TETRAD}} utilizes the theory of phylogenetic invariants to infer quartet trees from a SNP alignment. Species relationships are estimated for all quartet combinations of individuals, then all quartet trees are joined into a species tree with the software {\small{wQMC}} \autocite{Avni2015}. Quartet methods were designed to reduce computational time for tree-building with maximum likelihood methods \autocite{Ranwez2001}, making them particularly useful for SNP datasets such as these.

We ran {\small{TETRAD}} via Python 2.7.16 (www.python.org) for 50 samples of \textit{Ceratopteris}, and one sample of \textit{Acrostichum aureum} as the outgroup. Our input consisted of 51 individuals and 26891 SNPs; we ran 100 bootstrap iterations for the final consensus tree. As a comparison, we also ran {\small{RAxML}} v. 8.2.11 \autocite{Stamatakis2014} with 1000 bootstrap iterations.
% WDP: What was the exact call? 1000 rapid bootstrap (which means, I think, 200 additional full ML searches?) searched? 1000 independent, full ML searches?
We then plotted the final trees in R using the package phytools \autocite{Revell2012} and custom plotting functions.

\textbf{2.7. Polyploidy analysis}

Due to the presence of numerous cryptic species and morphological variation across the genus, we hypothesize there may be cryptic cytotype variation and/or ploidy type (allo vs. auto) within species, especially the polymorphic \textit{C. thalictroides}. Since a majority our specimens were from herbarium accessions, we could not perform flow cytometry or chromosome squashes. We utilized the R v. 3.5.2 \autocite{R_352} package gbs2ploidy to infer ploidy variation across samples. This package was designed to detect ploidy levels in variable cytoptype populations (specifically quaking aspen, \textit{Populus tremuloides}) using low-coverage (2X) genotyping-by-sequencing (GBS) or RADseq data; it was also tested on simulated data, and can detect diploids, triploids, or tetraploids \autocite{Gompert2017}. This package estimates ploidy from allele ratios, which are calculated from genome-average heterozygosity and bi-allelic SNPs isolated from a variant calling format (VCF) file, which is included in the output from \textit{ipyrad}.

To perform our analyses with gbs2ploidy, we first used the script vcf2hetAlleleDepth.py (github.com/carol-rowe666/vcf2hetAlleleDepth) to convert the VCF file produced by \textit{ipyrad} to the format needed for gbs2ploidy. We then ran gbs2ploidy, using the function estprops to estimate ploidy for each individual. We plotted the output from gbs2ploidy in R, with the mean posterior probability for allelic ratios on the \textit{y} axis and the 1:1, 2:1, and 3:1 ratios on the \textit{x} axis (see Fig. \ref{g2p}); we also included error bars for the 95\% equal tail probability intervals (ETPIs). We assigned ploidy to each individual using the  highest posterior mean estimate for a certain allelic proportion. If the ETPIs overlapped, we considered it an ambiguous assignment. We required sequencing depth to be 6X or greater during our data processing, which suggests that the ploidy assignments are likely accurate.

\textbf{2.8. Simulated RADseq analysis of the \textit{Ceratopteris richardii} genome}

Our sampling included relatively few individuals of \textit{Ceratopteris richardii}, so we utilized the published genome from \textcite{Marchant2019} to perform an \textit{in silico} RADseq digest, creating a pseudo-individual to include in our analyses. We used the program ddRADseqTools \autocite{Mora-Marquez2017} to digest the genome with the same enzymes used in our \textit{in vivo} digest (PstI and BfaI). Then we performed a simulated ddRADseq run with paired-end reads, demultiplexing, and trimming using ddRADseqTools. This simulated individual was added to our other samples, processed via the \textit{ipyrad} pipeline, and analyzed via STRUCTURE. 

\vspace{1cm}

{\large\textbf{3. Results}}

We retrieved an average of 2.58 x 10\textsuperscript{6} raw reads per sample (See Supplementary data 1). From the 60 herbarium samples and 36 silica-dried specimens, we retrieved an average of 5954 and 15029 loci, respectively. On average, we recovered nearly three times the number of loci from silica-dried specimens compared to the herbarium specimens. The 17 herbarium samples had been previously extracted before sending to the University of Wisconsin; the remaining samples (of both herbarium and silica-dried tissue) were extracted from leaf tissue at UW facilities. The UW facilities obtained much higher-quality extracted DNA than we did in our own lab, which contributed to some of the disparity between herbarium and silica-dried DNA quality.

Unfortunately, all samples of \textit{Ceratopteris richardii} (including \textit{Hnn}) either failed in sequencing, or had very poor sequencing quality. After an initial examination of the results of these samples, we chose to remove them from our final analyses. We did not want to attempt to make any inferences about this important species, nor the lab strain, with poor-quality data.
% WDP: Why are you making no reference to the supplementary materials here? Your results were robust either way - mention that here.

\textbf{3.1. Population and genomic structure analysis}

To determin the best \textit{K} value across {\small{STRUCTURE}} we combined several metrics. We used the best \textit{K} method by \textcite{Evanno2005}, which determined a \textit{K} value of 4. The best \textit{K} estimate described in the {\small{STRUCTURE}} manual \autocite{Pritchard2000} determined a \textit{K} value of 5. We also visually examined \textit{K} = 2 - 7 and determined that \textit{K} = 5 was the most biologically meaningful because it separated each named species, and also showed some intra-specific variability (Fig. \ref{structure}). Increasing \textit{K} past 5 did not add any meaningful population clusters. 

Named species of \textit{Ceratopteris} mostly clustered together at \textit{K} = 5 (Fig. \ref{structure}). \textit{Ceratopteris cornuta} and \textit{C. pteridoides} appeared as distinct populations, although there was some minor introgression from \textit{C. pteridoides} into the \textit{C. cornuta} population. \textit{Ceratopteris thalictroides} \textit{sensu latu} split in to three groups at \textit{K} = 5. The first group consisted of several individuals identified as \textit{C. thalictroides} (on herbarium sheets or in the field) that clustered with \textit{C. gaudichaudii}, a cryptic species of \textit{C. thalictroides} \autocite{Masuyama2010}. The remaining individuals of \textit{C. thalictroides} grouped into two populations, one consisting of mostly Old World individuals, and the other as entirely New World individuals. These two populations were strongly differentiated by {\small{STRUCTURE}}, being separated at all \textit{K} values.

\textbf{3.2. Species tree inference}

 The phylogeny resulting from our species quartet inference shows a similar pattern to the {\small{STRUCTURE}} output (Fig. \ref{phy}). Individuals of \textit{Ceratopteris pteridoides} form a monophyletic group, sister to the rest of the genus.z \textit{Ceratopteris cornuta} came out in two places in the phylogeny: sister to Old World \textit{C. thalictroides}, and in a grade outside the \textit{C. gaudichaudii}, Old World and New World \textit{C. thalictroides} clade. Our phylogeny grouped all individuals placed in the \textit{C. gaudichaudii} population by {\small{STRUCTURE}} in a monophyletic clade, sister to New World \textit{C. thalictroides}. 
 
 The {\small{RAxML}} tree shows a very similar structure to the {\small{TETRAD}} tree. The only major difference was that instead of New World \textit{C. thalictroides} and \textit{C. gaudichaudii} forming a monophyletic group, the former comes out as sister to all other species except \textit{C. pteridoides}. Graphical comparison of the two trees is included in Supplementary data 2.
 % WDP: Give a justification for these being in the supplementary materials; the usual one is "since these results are broadly similar we present them in the supplementary materials" or "since these results are qualitatively identical" or "do not affect conclusions" or some-such nonsense.

\textbf{3.3. Ploidy assignment across species}

\textit{Ceratopteris cornuta}, \textit{C. pteridoides}, and \textit{C. richardii} are known as a diploid (\textit{n} = 39) \autocite{Adjie2007, Hickok1977}; \textit{Ceratopteris thalictroides} and \textit{C. gaudichaudii} are known as tetraploid, with some cytotypic variation (\textit{n} = 77, 78) \autocite{Adjie2007, Masuyama2010}. Our ploidy analysis used allelic ratios to estimate the ploidy of samples. Diploids would have a 1:1 allele ratio in accordance with the two parental genomes present. Triploids (derived via hybridization) would have a 2:1 allele ratio, with two sets of chromosomes from a unreduced spore, and another from a haploid spore. Tetraploids could either have a 1:1 or 3:1 allele ratio: the former would be representative of an autopolyploid derived from a genome double event, or an allopolyploid with two sets of homeologous chromosomes; the latter would be representation of an allopolyploid that has undergone genomic restructuring  (i.e. compensated aneuploidy, \cite{Sigel2016}). 

Three of four samples identified as \textit{Ceratopteris cornuta} showed relatively equal allelic proportions of 1:1 and 3:1, indicating a possible tetraploid; the fourth sample was ambiguous. Some samples of the known diploid \textit{C. pteridoides} showed a similar pattern, with one putative diploid and three tetraploids. However, both species are known as a diploid so this may be a mis-assignment of ploidy by our analysis, or detecting of some interesting genomic structure. \textit{Ceratopteris gaudichaudii} had a high 1:1 allele ratio, suggesting that it is an allotetraploid. \textit{Ceratopteris thalictroides} is also known to be a tetraploid \autocite{LloydTax1974, Masuyama2010}, and individuals in our analysis were estimated to have 1:1, 2:1, and 3:1 allele ratios. Despite this variation, there was a distinct pattern between Old World and New World individuals: Old World \textit{C. thalictroides} was most commonly had a 1:1 allele ratio (autopolyploid), whereas New World \textit{C. thalictroides} had  3:1 allele ratios (allopolyploid). 

Mis-assignment of ploidy could be due to coverage being at the lower end of the scale needed for true ploidy assignments, but it also could indicate hybridization leading to genomic complexity across species. Including chromosome counts and flow cytometry data in future analyses would help better understand the potential variation in polyploidy origins of \textit{C. thalictroides}.

\textbf{3.4. Analysis of simulated RADseq data}

We attempted to create a pseudo-individual to include in our analysis using the \textit{Ceratopteris richardii} genome assembly, and we were able to obtain 9,047,554 simulated raw sequences from our \textit{in silico} RADseq digest. These reads, in combination with the rest of our \textit{in vivo} data, were processed with the \textit{ipyrad} pipeline. The \textit{ipyrad} pipeline consists of 7 steps, each performing one aspect of RADseq data processing \autocite{Eaton2020}. Our pseudo-individual of \textit{C. richardii} performed nearly identically to all of the other samples until Step 6. Here, similar sequences are clustered across samples, and then these shared sequences are mapped to a reference (our reference was a \textit{de novo} assembly, because the quality and coverage of the \textit{C. richardii} genome was too poor to serve as a reference).
% WDP: Condense the above; this is too much background (and some of this is methods, I sense, so if you really can't cut it move it to the methods).
This requires that all individuals are related relatively close to one another, so that they share a large number of sequences. Very few sequences were retained for our pseudo-individual of \textit{C. richardii}, indicating that the sequences obtained are too different from the \textit{in vivo} samples. In order to retain enough SNPs to run STRUCTURE, we required that only 3 individuals needed to have data at a given locus for it to be included in the final data set. In our entirely \textit{in vivo} dataset, this minimum number of individuals was 30; we designated such a high number to reduce missing data. In our dataset with the pseudo-individual of \textit{C. richardii}, a vast majority of our retained loci had missing data.

The differences between the pseudo-individual derived from the \textit{Ceratopteris richardii} and our other samples is mostly likely due to the fact that the current assembly (v. 1.1) only represents about one third of the total genome, and consists of mostly short reads. Long reads that can span the many repetitive elements in the genome comprise only 0.03\% of the final assembly \autocite{Marchant2019}. Because of these factors, the current genome assembly contains a small amount of short sequences that are difficult to accurately compare to the fragments generated by \textit{in vivo} RADseq. We have included the results from our \textit{in silico} RADseq digest in Supplementary data 2, but do not present them in the main manuscript.
% WDP: SAY WHAT THE RESULTS SHOW - they don't change your conclusions, so make that clear. You're making it sound like you're hiding something when, in fact, running all of this through makes your conclusions even clearer.

\vspace{1cm}

{\large\textbf{4. Discussion}}

To the best of our knowledge, this is the first application of next-generation sequencing techniques to the genus \textit{Ceratopteris}. Using population genomic, phylogenetic, and ploidy inferences, we recover similar species relationships found in previous studies, find support for \textit{C. gaudichaudii} as a genetically distinct cryptic species, identify a novel cryptic lineage from within \textit{C. thalictroides} \textit{sensu latu} in Central and South America, and show strong genetic variation across the broad range of \textit{C. coruta}.

Not included in our results are specimens of the model fern \textit{Ceratopteris richardii}, including the lab strain \textit{Hnn}. All individuals samples we had of this species yielded very poor sequencing results. We felt it was appropriate to remove these samples because we did not want to try and infer anything about this important species with poor quality data. Future work will be aimed at including wild collections of \textit{C. richardii} as well as the lab strain \textit{Hnn} to determine the relationship of this species to the rest of the genus.

\textbf{4.1. Species boundaries in \textit{Ceratopteris}} 

\textbf{4.1.1. Variation within \textit{Ceratopteris cornuta}}

\textit{Ceratopteris cornuta} is found from western Africa to Northern Australia. The sterile fronds are consistently deltoid to lanceolate in shape, and pinnate to bipinnate in dissection \autocite{LloydTax1974}. Our population genomic analyses distinguished all specimens of \textit{C. cornuta} as a distinct population (at \textit{K} greater than 4, Fig. \ref{structure}). All samples of \textit{C. cornuta} grouped together at \textit{K} = 2 - 7, along with one sample of \textit{C. thalictroides} from Nepal (\textit{Fraser-Jenkins 1564}) and an unidentified sample from Brazil (\textit{L. Camargo de Abreu 21}). 

Interestingly, our population genomic analysis and phylogenetic reconstruction show that individuals of \textit{C. cornuta} from the western portion of its range (Sierra Leon, Brazil) are genetically different from individuals in the eastern portion of its range (Tanzania, Oman, Nepal) (Fig. \ref{structure, phy}). At \textit{K} = 6, individuals from Sierra Leon (Cr04) and Brazil (Sp01) are separated slightly from the other three individuals of \textit{C. cornuta}, and contain some genetic influence that is not attributed to another sampled species. In addition, this small clade is placed as sister to the clade containing \textit{C. thalictroides sensu latu} and \textit{C. gaudichaudii} (Fig. \ref{phy}) It is important to note that there are no known collections of \textit{C. cornuta} from the Americas. However, \textit{Ceratopteris richardii} has been found in western Africa \cite{LloydTax1974}. We may have sampled African and South American individuals of \textit{C. richardii}. Similar patterns of Africa-South American gene flow have been seen in other broadly distributed pteridophytes. Recent gene flow between the two continents was observed in cosmopolitan \textit{Pteridium} \autocite{Wolf2019}, phylogenetic reconstruction has shown multiple migrations of grammitid ferns from South America to Africa \autocite{Sundue2014}, and several species (e.g. \textit{Asplenium monanthes}, \textit{Polypodium polypodioides} )have a disjunct distribution between the two continents \cite{kornas1993}. Therefore, it is possible that there could be some trans-Atlantic gene flow in \textit{Ceratopteris richardii}. Unfortunately, we do not have a known sample of \textit{C. richardii} to confidently compare to this group. When we did include samples of \textit{C. richardii} (from an herbarium specimen and the simulate digest of the \textit{C. richardii} genome) in a {\small{STRUCTURE}} analysis, they were placed roughly in the same population as \textit{C. cornuta} (Supplementary data 2). As we have mentioned before, these data were poor quality and low coverage. While it is a window into potential relationships, increased sampling is needed to draw any meaningful conclusions. 

\textbf{4.2.2. Genomic conservation in \textit{Ceratopteris pteridoides}}

Endemic to the New World tropics, \textit{Ceratopteris pteridoides} has perhaps the most unique morphology of the genus: sterile fronds are simple, palmately or pinnately lobed, with enlarged, air-filled stipes (petioles). They are most commonly found unrooted and floating on the surface of slow-moving water \autocite{LloydTax1974}. \textit{Ceratopteris pteridoides} is also very strongly differentiated by the {\small{STRUCTURE}} analysis, with samples identified as \textit{C. pteridoides} showing little to no introgression from other populations. Influence from \textit{C. pteridoides} does appear in \textit{C. cornuta}, \textit{C. gaudichaudii}, and some samples of \textit{C. thalictroides}. \textit{Ceratopteris pteridoides} is not known outside of the neotropics, and so it is difficult to say whether this is true introgression into other species, or if it is an artifact of slightly lower sequencing coverage in these individuals. It is known, however, that \textit{C. pteridoides} readily hybridizes with \textit{C. richardii} where their ranges overlap in northern South America \autocite{hickok1974}. Including the latter in future work may help to reveal the extent to which natural hybridization is occurring between diploid \textit{C. pteridoides} and \textit{C. richardii} in the New World. 

\textbf{4.1.3. Cryptic species of \textit{C. thalictroides}}

There was one named cryptic species of \textit{C. thalictroides} included in our study, \textit{Ceratopteris gaudichaudii}, which has two varieties: \textit{C. g. gaudichaudii} and \textit{C. g. vulgaris} \autocite{Masuyama2010}; only the latter was included in our analysis. \textit{Ceratopteris g. vulgaris} can be found throughout the Pacific in Japan, the Philippines, Hawaii, Guam, Taiwan, Southeast Asia, and Northern Australia; \textit{Ceratopteris g. gaudichaudii} is narrowly endemic to Guam \autocite{Masuyama2010}. Individuals identified as \textit{C. thalictroides} from Taiwan, Hawaii, Japan, China, and Australia were identified via our population genomic and phylogenetic analyses as being \textit{C. gaudichaudii}, although it is impossible to tell from our limited sampling which sub-species they align with. The individual from Australia was examined and keyed out morphologically to \textit{C. g. vulgaris}, using the key from Masuyama and Watano (2010). This same individual was growing a few meters away from an individual of Old World \textit{C. thalictroides}, indicating that these two morphologically similar species are growing sympatrically but with limited to no gene flow. 

All of the samples from Hawaii included in this study show alignment to \textit{C. gaudichaudii} in our population structure analysis. \textcite{Wagner1950} hypothesized that \textit{Ceratopteris} was not native to Hawaii, and may have been introduced from Asia either as a food source or accidentally as a weed in taro patches. In addition, work by \textcite{Hickok1979} supported Hawaiian \textit{Ceratopteris} as a distinct species from \textit{C. thalictroides}. Japan was hypothesized to be the source of Hawaiian \textit{Ceratopteris} by \textcite{Lloyd1973}. This is plausible since \textit{Ceratopteris gaudichaudii} is relatively common in taro patches in Japan, and could have been brought to Hawaii via settlers or trade. Interestingly, one sample from Hawaii was shown to be a potential hybrid between \textit{C. gaudichaudii} and \textit{C. thalictroides}. The specimen (\textit{L. M. Crago 2005-058} US) is highly dissected and has may proliferous buds, which could indicate a hybrid or perhaps just an older plant. This sample was also placed as sister to Old World \textit{C. thalictroides} in our TETRAD and RAxML trees. Hybrid ancestry would explain grouping with \textit{C. gaudichaudii} in one analysis, and \textit{C. thalictroides} in another.

Two named cryptic species of \textit{C. thalictroides} were not included in our analysis: \textit{C. oblongiloba} and \textit{C. froseii}. We were not able to request type material, we had to rely on our herbarium sampling of their ranges to attempt to include these species. \textit{Ceratopteris oblongiloba} is found throughout Cambodia, Indonesia (Sumatra and Java), the Philippines (Luzon), and Thailand (Malay) \autocite{Masuyama2010}. Our only specimen from this range is from the Philippines, does not key to \textit{C. oblongiloba}, and aligns entirely with Old World \textit{C. thalictroides}. This either indicates that we did not unknowingly sample \textit{C. oblongiloba}, or it has a lesser degree of genomic distinction than the other cryptic species, \textit{C. gaudichaudii}. \textit{Ceratopteris froesii} is a Brazilian endemic characterized by very small fronds: about 4 cm or less for fertile leaves \autocite{Masuyama2010}. We only had four accessions from Brazil, and none were small plants. Inclusion of this unique species in future studies may help us further understand New World diversity of \textit{Ceratopteris}.

\textbf{4.1.4. \textit{Ceratopteris thalictroides, sensu latu}}

Individuals of \textit{Ceratopteris thalictroides} that did not group with \textit{C. gaudichaudii} break in to two groups: an Old World and a New World clade. These two clades are separated at all values of \textit{K} in our {\small{STRUCTURE}} analysis (Fig. \ref{structure}) and are paraphyletic (Fig. \ref{phy}), indicating strong genetic, if not morphological, separation.

The Old World \textit{C. thalictroides} clade includes mostly individuals from Asia and Australia, but also a few from South America. Increasing \textit{K} in {\small{STRUCTURE}} did not break up this large group, suggesting that there is recent gene flow across the Pacific via long distance dispersal or human-mediated transport. Several fern and angiosperm genera share such a disjunct distribution between eastern Asia and North to Central America, either by vicariance or long distance dispersal \autocite{Kato1983, Les2003}. One individual from Hawaii (\textit{L. M. Crago 2005-058} US) shows shared ancestry with \textit{C. gaudichaudii} and Old World \textit{C. thalictroides}, indicating that trans-Pacific gene flow might be possible in \textit{Ceratopteris}. Another individual from Japan shows a similar pattern, but with a much larger proportion of its genome comprised of the Old World \textit{C. thalictroides} population. There may be continued gene flow not only within Old World \textit{C. thalictroides}, but between \textit{C. thalictroides} and \textit{C. gaudichaudii}. The Hawaiian archipelago, as well as other Pacific islands, have been hypothesized to be a stepping-stone for trans-Pacific dispersal in angiosperm taxa \autocite{Harbaugh2009, Wright2001}, a theory that could also hold in ferns, which are known to have very high dispersal capabilities \autocite{Barrington1993, Tryon1970}. 

New World \textit{C. thalictroides} is strongly differentiated from Old World \textit{C. thalictroides} in our population structure analysis, and shows virtually no gene flow between any other species at \textit{K} greater than 3. This strong distinction between Old and New World is present in the phylogeny as well: New World \textit{C. thalictroides} is placed in a clade as sister to \textit{C. gaudichaudii}, and this clade is in turn sister to Old World \textit{C. thalictroides} (Fig. \ref{phy}). \textit{Ceratopteris gaudichaudii} has been hypothesized to be an allopolyploid hybrid \autocite{Adjie2007}; perhaps the derivation of its sister species, New World \textit{C. thalictroides}, could have been from a hybridization event with one of the parents of \textit{C. gaudichaudii}, or genome doubling to form an autotetraploid, followed by long distance dispersal. Our ploidy inference suggests that either form of polyploidy is possible, but slightly more individuals of New World \textit{C. thalictroides} appeared to be of allopolyploid origin, potentially with subsequent genomic rearrangement (Fig. \ref{g2p}).

\textbf{4.2. Origins of cryptic species}

All of the cryptic species in the \textit{Ceratopteris thalictroides} complex are known to be tetraploid (\textit{n} = 77, 78; \cite{Masuyama2010}). Several authors have proposed that different cryptic species have been derived via differing mechanisms of polyploidy (i.e., allo- vs. autopolyploid). \textcite{Adjie2007} identified Asian \textit{C. thalictroides} \textit{sensu latu} as paraphyletic and hypothesized that \textit{C. thalictroides sensu strictu} and the cryptic species \textit{C. gaudichaudii} and \textit{C. oblongiloba} are allopolyploid hybrids between \textit{C. cornuta} and unnamed, potentially extinct, diploid progenitors. \textcite{McGrath1994} suggested that New World \textit{C. thalictroides} could either be an autopolyploid, or an allopolyploid hybrid between \textit{C. richardii} and another diploid species.

Our population structure analysis do not reveal any cryptic species of \textit{C. thalictroides} as distinct allotetraploids (i.e., containing roughly equal genomic proportions from two parents). Each cryptic species (Old World \textit{C. thalictroides}, New World \textit{C. thalictroides}, and \textit{C. gaudichaudii}) were represented as separate populations, sharing little to no genetic material with any other population (Fig. \ref{structure}). We were not able to include the Asian cryptic species \textit{C. oblongiloba}, and so cannot say whether or not it is of recent hybrid origin. 

Our ploidy estimation, however, indicates that the Old World clade of \textit{C. thalictroides} and \textit{C. gaudichaudii} may be a balanced allopolyploid, or autotetraploid. This former supports the theory proposed by \textcite{Adjie2007}, but the fact that there is no gene flow between Old World \textit{C. thalictroides} and any other species supports the latter. The New World clade of \textit{C. thalictroides} may be of allopolyploid origin. In our {\small{TETRAD}} phylogeny, it is sister to \textit{C. gaudichaudii}; however, in the {\small{RAxML}} phylogeny it is sister to a clade containing Old World \textit{C. thalictroides}, western \textit{C. cornuta}, and \textit{C. gaudichaudii} (See Supplementary data 2). New World \textit{C. thalictroides} was the only clade to come out in two different places in our two phylogenetic reconstructions. An allopolyploidy origin might explain this, with one method placing New World \textit{C. thalictroides} closer to the maternal progenitor, and the other placing it closer to the paternal progenitor.

{\textbf{4.3. Evolutionary drivers in \textit{Ceratopteris}}}

\textbf{4.3.1. Biogeography and reproductive boundaries}

Weak reproductive boundaries are a theme across ferns, evidenced by numerous hybrids and reticulate species complexes (\emph{e.g.}, \cite{Barrington1989,  Paris1989, Sessa2012a, Sigel2016, Yatabe2009}), and intergeneric hybridization across 60 MY of evolution \autocite{Rothfels2015}. Although ferns do have some pre- and post-zygotic barriers \autocite{Haufler2016}, they are not as strong as in angiosperms (e.g., \cite{De_Nettancourt1997, Lafon-Placette2016}). In addition, fern spores are very easily dispersed \autocite{Tryon1970, Smith1972}, which limits isolation by distance. 

In the case of \textit{Ceratopteris}, its large range may be influenced by the Intertropical Convergence Zone (ITCZ) (see \cite{Dettmann1992, LloydTax1974, Schneider2014}). The ITCZ is created by the rising portion of the Hadley Cell circulation in both hemispheres; it moves with the seasons, following the thermal equator \cite{Schneider2014}. It has been hypothesized as a possible mechanism of transport of Asian ferns to Hawaii \autocite{Geiger2007}, and could carry spores great distances from one area of the tropics to another, provided they make it high enough into the atmosphere, perhaps by monsoons or hurricanes. Continuous dispersal and/or movement of propagules is recognized as an important factor in widespread angiosperm hydrophytes \autocite{Les2003, Spalik2014}. The ITCZ could be a means of dispersal for \textit{Ceratopteris}, which would help explain the continued gene flow across the large range of its species such as Old World \textit{C. thalictroides}, or \textit{C. cornuta}. 

Even with a high level of dispersal, there is potential evidence for reproductive boundaries and sympatric speciation in \textit{Ceratopteris}. Two specimens collected meters apart in Australia keyed to \textit{C. thalictroides sensu strictu} and \textit{C. gaudichaudii} var. \textit{vulgaris}. In addition, specimens of Old World \textit{C. thalictroides} and New World \textit{C. thalictroides} were obtained from similar regions in South America. Sympatric cryptic species are known from other genera of ferns \autocite{Patel2019, Yatabe2009}, lichens \autocite{Del_Carmen_Molina2011}, angiosperms \autocite{Les2015, Soltis2007}, as well as in animals \autocite{Amor2014, Hebert2004, Nygren2014}. Different polyploid origins in \textit{Ceratopteris} could potentially be the cause of sympatric speciation, with allo- and autotetraploids becoming isolated from one another. Another mechanism of differentiation could be minor ecologically separation. All \textit{Ceratopteris} species occupy very similar habitats \autocite{LloydTax1974, Masuyama2010}. However, in the case of the two Australian specimens collected at the same locality, the individual of Old World \textit{C. thalictroides} was growing in full sun, while \textit{C. gaudichaudii} was growing in deep shade. The former is potentially an autotetraploid (see Fig. \ref{g2p}), whereas the latter may be an allotetraploid of hybrid origin \autocite{Adjie2007}. How reproductive boundaries have evolved between these co-occurring cryptic species area of future work.

\textbf{4.3.2. Morphology and ecology}

One of the most confusing aspects of \textit{Ceratopteris} is the disparity between morphological and genetic distinction between species. The three diploid species are relatively morphologically distinct \autocite{LloydTax1974}, especially in comparison to the cryptic species complex of \textit{C. thalictroides sensu latu}. However, all the diploid species in the genus are known to hybridize with one another \autocite{hickok1974, Hickok1977, LloydTax1974}. In contrast, \textit{C. thalictroides} and associated cryptic species are much more similar morphologically, yet different populations (some named as cryptic species) can be almost entirely reproductively isolated \autocite{Hickok1979, Masuyama2002, Masuyama2010}. However, Throughout the genus, however, the body plan, life history, and niche requirements of all species are quite similar, indicating limited ecological divergence.

A potential explanation for morphological stasis in \textit{Ceratopteris} relies on the influence of ecological pressures \autocite{Bickford2007}. For a fern, \textit{Ceratopteris} has an unusually short life cycle of only about four months from spore to spore-bearing adult \autocite{Stein1971}. It grows in temporary water sources such as pond edges, swamps, or tarot patches \autocite{LloydTax1974}. Because of its ephemeral habitat, there are likely selective pressures for maintaining morphological stasis and rapid generation times. Fossil spore and leaf evidence also support morphological stasis in the Ceratopteridoideae dating back about 47 MY \autocite{Dettmann1992, Rozefelds2016-dr}. Fossil leaf impressions of the extinct genus \textit{Tecaropteris} from Australia are very similar morphologically to extant \textit{Ceratopteris} \autocite{Rozefelds2016-dr}. Australia has become increasingly dry and seasonal since the Eocene, when \textit{Tecaropteris} lived \autocite{Rozefelds2016-dr, McKenna2010}. Today, \textit{Ceratopteris} is exceedingly difficult to find during the dry season in northeastern Australia, but plentiful during the wet season, even growing as a weed in garden ponds (Dr. Ashley Field, personal communication). Almost all leaves of \textit{Ceratopteris} species are very thin, delicate, and herbaceous \autocite{LloydTax1974}. When removed from the plant they wilt into an unrecognizable state within minutes (SPK, personal observation). These low-energy investment leaves can be grown quickly and without an excess of resources from the plant \autocite{Reich2014, Wright2004}, potentially as a result of the fast life cycle employed by \textit{Ceratopteris}; any major shift in leaf development might inhibit \textit{Ceratopteris} from proceeding through its life cycle before its water source dries up. Therefore, despite the genetic diversity in the genus, morphological diversity has changed very little due to niche constraints. 

\vspace{1cm}

{\large\textbf{5. Conclusion}}

In summary, our findings suggest that there may be more cryptic species of \textit{Ceratopteris thalictroides} yet to be discovered, especially in the New World. The potential for cryptic species in the Neotropics has been noted by several authors \autocite{Masuyama2010, LloydTax1974}, and this area of the world remains one of the least studied parts of the range of \textit{Ceratopteris}. Considering that this is where the model species \textit{C. richardii} is native, more work on all species of the genus in the New World is especially warranted. Researching \textit{Ceratopteris} in its natural environment is critical to inform future lab studies on \textit{C. richardii}, and the potential incorporation of other species for comparative studies. Most importantly, understanding speciation and cryptic species within \textit{Ceratopteris} is important for its use as model system, as well as increasing our knowledge of evolutionary processes in ferns and across vascular land plants.

\vspace{1cm}

{\large\textbf{Acknowledgements}}

The authors thank the University of Wisconsin Biotechnology Center DNA Sequencing Facility for providing DNA extraction, library prep, and DNA sequencing facilities and services. We would also like to thank the University of Utah Center for High-Performance Computing, particularly Anita Orendt, for providing computational resources for data analyses. Thank you to Ashley Field and Tzu-Tong Kao helping with field work; Jacob S. Suissa and David Barrington for sending herbarium specimens; and Ryan Choi and Jacob S. Suissa for their thoughtful comments and edits to the manuscript. SPK is funded by a National Science Foundation Graduate Research Fellowship and the Joseph E. Greaves Endowed Scholarship from Utah State University. WDP and the Pearse lab are funded by National Science Foundation ABI‐1759965, NSF EF‐1802605, United States Department of Agriculture Forest Service agreement 18‐CS‐11046000‐041, and UK NERC grant NE/V009710/1.

\vspace{1cm}

{\large\textbf{Appendix A. Supplementary material}}

Supplementary data 1.

Supplementary data 2: Simulated data

\end{flushleft}
\vspace{30cm}

\begin{figure}[H]
\centering
\includegraphics[scale=0.45]{figure1.pdf}
\caption{Collection localities for samples of \textit{Ceratopteris} included in the present study. These collections roughly indicate the range for each species. Not all accessions are included on this map because many herbarium specimens did not have GPS coordinates, or several samples were from nearly identical locations. Localities of \textit{C. richardii} are those of sampled individuals, but were ultimately not included in our analyses.}
\label{map}
\end{figure}

\begin{figure}[H]
\centering
\includegraphics[scale=1]{g2p.pdf}
\caption{Ploidy estimates for a subset of individuals. The X axis displays the estimated allele ratios (1:1, 2:1, 3:1), and the Y axis shows the Bayesian posterior proportion of allelic ratios; 95\% credible intervals are shown on each bar. The allelic ratio with the highest posterior estimate was used to assign ploidy to an individual. If the credible intervals overlapped, ploidy was assigned as ambiguous. \textit{Ceratopteris cornuta} (green) and \textit{C. pteridoides} (purple) are both known diploids (\textit{n} = 39; \cite{Hickok1977}). All specimens of these species had a 3:1 allele ratio, indicating potential homoploid hybridization, or genomic restructuring within diploids to yield such unbalanced allele ratios. \textit{C. gaudichaudii} and \textit{C. thalictroides} are known tetraploids (\textit{n} = 77, 78; \cite{Masuyama2010}). \textit{Ceratopteris thalictroides} 1 (grey) had two distinct patterns of allele ratios, a 1:1 (autopolyploid) group and a mixed to ambigious group. The former is entirely from the Old World, whereas the latter group contains individuals from the Old and New Worlds. \textit{Ceratopteris thalictroides} 2 (yellow) is only found in the New World, and all individuals had mixed 1:1 and 3:1 (allopolyploid) allele ratios. See Supplementary data 2 for sample IDs and all ploidy estimates.}
\label{g2p}
\end{figure}

\begin{figure}[H]
\centering
\includegraphics[scale=0.6]{k2-6_species.pdf}
\caption{{\small{STRUCTURE}} plots for \textit{K} = 2 - 6. Each bar represents an individual, and each color is representative of a single genomic population. A single bar with multiple colored segments indicates ancestral admixture from \textit{K} source populations. \textit{K} = 5 was determined to be the most biologically informative. At \textit{K} = 5, all named species of \textit{Ceratopteris} fall into distinct populations. \textit{Ceratopteris thalictroides} is split into three groups: Old World (yellow), New World (blue), and individuals grouping with \textit{C. gaudichaudii} (green). \textit{Ceratopteris cornuta} is the only species with evidence of ongoing or recent admixture, most evident with \textit{C. pteridoides}. At \textit{K} = 6, the yellow bars may be indicative of \textit{C. richardii}, but we not have a sample here to compare it to; see Supplementary data 2.}
\label{structure}
\end{figure}

\begin{figure}[H]
\centering
\includegraphics[scale=0.65]{figure4.pdf}
\caption{Phylogeny of \textit{Ceratopteris} generated via quartet methods. The tree is rooted with the sister genus to \textit{Ceratopteris}, \textit{Acrostichum}. \textit{Ceratopteris pteridoides} is the most basal clade of the genus, and a clade of \textit{C. cornuta} is the next to diverge. This clade might actually be a sampling of \textit{C. richardii}; however, future sampling is needed to confirm this. Several individuals originally identified as \textit{C. thalictroides} grouped with \textit{C. gaudichaudii}, indicating that this cryptic species is indeed prolific. The othe individuals of \textit{Ceratopteris thalictroides sensu latu} breaks into two groups here: an Old World clade (1) and a New World clade (2). A clade of \textit{C. cornuta} is sister to Old World \textit{C. thalictroides} 1, which supports it as a potential progenitor of the latter species \autocite{Adjie2007}.}
\label{phy}
\end{figure}

\vspace{30cm}
\printbibliography
\end{document}
